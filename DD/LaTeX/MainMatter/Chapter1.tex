% ------------------------------------------------------------------------ %
% !TEX encoding = UTF-8
% !TEX TS-program = pdflatex
% !TEX root = ../Project.tex
% !TEX spellcheck = en-EN
% ------------------------------------------------------------------------ %
%
% ------------------------------------------------------------------------ %
% 	CHAPTER TITLE
% ------------------------------------------------------------------------ %
%
\chapter{Introduction}
%
\label{cap:introduction}
%
% ------------------------------------------------------------------------ %
%
This document is the Design Document (DD) of a mobile application called Travlendar+. It is mainly addressed to the software development team and its purpose is to provide an overall view of the architecture of the new system. The document defines: 
\begin{itemize}
\item	High level architecture;
\item	Used design patterns;
\item	Main components of the system;
\item	Runtime behavior.
\end{itemize}
%
% ------------------------------------------------------------------------ %
%
\section{Description of the given problem}
Travlendar+ is a mobile, calendar-based application that helps the user to manage his appointments and to a greater extent set up the trip to his destination, choosing the best means of transport depending on his needs. \\
Travlendar+ will choose the most suitable way to get the user to his destination between a large pool of options, considering public transportation, personal vehicles, locating cars or bikes of sharing services and walking to the destination. It will take account of weather, traffic, possible passengers if any, the user-set break times and the potential will to minimize the carbon footprint of the trip, always focusing on taking him on time to his scheduled appointments. \\
Eventually the user will be able to purchase the tickets he will use to reach his destination in-app. The great customizability is one of the main strengths of Travlendar+, being able to fully comply with the user needs. 
%
% ------------------------------------------------------------------------ %
%
\section{Definitions and Acronyms}
%
\subsection{Definitions}
List of the definitions used in this paper:
\begin{itemize}
\item	Reverse Proxy: the component that will send the information received from multiple servers to the correct client.
\item	ToTrip: often used in code to refer to a trip with the event ad destination;
\item	FromTrip: often used in code to refer to a trip with the event ad departure location;
\item	Primary Event: feasible event;
\item	Secondary Event: not feasible event;
\item	Dynamic Event: event with no to/from trips and beginning changing on various factors.
\end{itemize}

\subsection{Acronyms}
List of the acronyms used in this paper:
\begin{itemize}
\item	RASD: Requirements analysis and specification document;
\item	DD: Design document;
\item	MOT: Means of transport;
\item	RMI: Remote Method Invocation;
\item	API: Application Programming Interface;
\item	UX: User experience;
\item	BCE: Boundary Control Entity;
\item	UI: User interface;
\item	IaaS: Infrastructure as a Service;
\item	REST: Representational State Transfer;
\item	JNDI: Java Naming and Directory Interface;
\item	JDBC: Java DataBase Connectivity;
\item	EJB: Enterprise Java Beans;
\item	DBMS: Data Base Management System;
\item	JEE: Java Enterprise Edition;
\item	EIS: Enterprise Information System;
\item	JSP: Java Server Pages;
\item	JSF: Java Server Faces;
\item	JPA: Java Persistence API.

\end{itemize}
%
% ------------------------------------------------------------------------ %
%
\section{Revision History}
\begin{itemize}
\item	26/11/2017 Version 1.0.0 - First complete drawing up of the DD;
\item	30/11/2017 Version 1.1.0 - Runtime view modifications;
\item	07/01/2018 Version 1.2.0 - Added descriptions to Runtime view diagrams.
\end{itemize}
%
% ------------------------------------------------------------------------ %
%
\section{References}
Documents list:
\begin{itemize}
\item Mandatory Project Assignments.pdf
\end{itemize}
Online sites list:
\begin{itemize}
\item	https://developers.google.com/maps/ (used also to design mockups)
\item	https://developers.google.com/google-apps/calendar/ (used also to design mockups)
\item	https://openweathermap.org/api
\item	https://www.interoute.it/what-iaas
\item	http://www.cloudcomputingpatterns.org
\item	https://www.ibm.com/support/knowledgecenter/en/SSZLC2\textunderscore7.0.0/
\item	com.ibm.commerce.developer.doc/concepts/csdmvcdespat.htm (MVC)
\end{itemize}
%
% ------------------------------------------------------------------------ %
%
\section{Document Structure}
The paper is structured as follows:
\begin{itemize}
\item	Chapter 1: Introduction to the document, including some information
about its composition and the description of the problem;
\item	Chapter 2: This chapter shows the main components of the system and the relationships between them. It also explains the main architectural styles and patterns adopted in the design of the system;
\item	Chapter 3: This chapter explains how the system will work using algorithms. Java code is used to write down the most significant algorithms for the application;
\item	Chapter 4: This chapter shows mockups of the application and more details about the User Interface using UX and BCE diagrams;
\item	Chapter 5: This chapter explains how the decisions taken in the RASD are associater to design decisions;
\item	Chapter 6: This chapter describes the order of implementation of the subcomponents of the system and the order in which integrate
such subcomponents and test the integration;
\item	Chapter 7: Effort spent by the authors to draw up the document.
\end{itemize}
%
% ------------------------------------------------------------------------ %
%
\section{Used tools}
The tools used to create this document are:
\begin{itemize}
\item	Photoshop for mockups;
\item	Draw.io for diagrams;	
\item	Atom for algorithms drawing up;
\item	Github as version controller and to share documents;
\item	LaTeX for typesetting this document;
\item	Texmaker as editor;
\end{itemize}
%
% -----------------------------END------------------------------------- %
