% ------------------------------------------------------------------------ %
% !TEX encoding = UTF-8
% !TEX TS-program = pdflatex
% !TEX root = ../Project.tex
% !TEX spellcheck = en-EN
% ------------------------------------------------------------------------ %
%
% ------------------------------------------------------------------------ %
% 	CHAPTER TITLE
% ------------------------------------------------------------------------ %
%
\chapter{Introduction}
%
\label{cap:introduction}
%
% ------------------------------------------------------------------------ %
%
Intro.
%
% ------------------------------------------------------------------------ %
%
\section{Description of the given problem}
Travlendar+ is a mobile, calendar-based application that helps the user to manage his appointments and to a greater extent set up the trip to his destination, choosing the best means of transport depending on his needs. \\
Travlendar+ will choose the most suitable way to get the user to his destination between a large pool of options, considering public transportation, personal vehicles, locating cars or bikes of sharing services and walking to the destination. It will take account of weather, traffic, possible passengers if any, the user-set break times and the potential will to minimize the carbon footprint of the trip, always focusing on taking him on time to his scheduled appointments. \\
Eventually the user will be able to purchase the tickets he will use to reach his destination in-app. The great customizability is one of the main strengths of Travlendar+, being able to fully comply with the user needs. 
%
% ------------------------------------------------------------------------ %
%
\section{Definitions and Acronyms}
%
\subsection{Definitions}
\subsection{Acronyms}
%
% ------------------------------------------------------------------------ %
%
\section{Revision History}
%
% ------------------------------------------------------------------------ %
%
\section{References}
Documents list:
\begin{itemize}
\item Mandatory Project Assignments.pdf
\end{itemize}
%
% ------------------------------------------------------------------------ %
%
\section{Document Structure}
%
% ------------------------------------------------------------------------ %
%
\section{Used tools}
%
% -----------------------------END------------------------------------- %
