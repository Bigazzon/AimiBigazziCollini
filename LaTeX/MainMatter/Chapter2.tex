% ------------------------------------------------------------------------ %
% !TEX encoding = UTF-8
% !TEX TS-program = pdflatex
% !TEX root = ../Project.tex
% !TEX spellcheck = en-EN
% ------------------------------------------------------------------------ %
%
% ------------------------------------------------------------------------ %
% 	CHAPTER TITLE
% ------------------------------------------------------------------------ %
%
\chapter{Overall Description}
%
\label{cap:overalldescription}
%
% ------------------------------------------------------------------------ %
%
\section{Product Perspective}
Travlendar+ will be developed as a mobile application that relies on the use of Google maps and Google calendar APIs. \\
Its user interface will be composed by two main tabs, one with a calendar, to schedule user's events and the other one with a map to manage the movements of the user. \\
In the future will have a service of technical assistance via chat. \\
The application will not provide any API for integration with other systems.
\\
\\
**************************************************************************************************************************************************************************** \\
Further details on the shared phenomena and a domain model (class diagrams and statecharts)
%
% ------------------------------------------------------------------------ %
%
\section{Product Functions}
******************************************************************************************************************************************** \\
Requirements
%
% ------------------------------------------------------------------------ %
%
\section{User Characteristics}
The user of the system-to-be is every person who wants to schedule appointments in a calendar and manage his movements from a location to another at the same time.
Users can use it to organize work events, but also include family or spare time events. The application doesn't have any age limit, or any other restriction applied to the user characteristic. In order to make the application work without limitation the user need to have access to the Internet, but he can access and modify the calendar offline.
%
% ------------------------------------------------------------------------ %
%
\section{Assumptions, Dependencies, Constraints}
**************************************************************************************************************************************************************************** \\
Domain assumptions
\\
Regulatory policy:
The System asks the User for the permission to acquire, store and use his personal data, and informs him that won't take any responsibility for a use of it that doesn't complies with the local laws and policies, by means of the User agreement's acceptance.
The System under request of the User must delete all his personal data.
%
% -----------------------------END------------------------------------- %