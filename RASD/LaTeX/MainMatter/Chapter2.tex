% ------------------------------------------------------------------------ %
% !TEX encoding = UTF-8
% !TEX TS-program = pdflatex
% !TEX root = ../Project.tex
% !TEX spellcheck = en-EN
% ------------------------------------------------------------------------ %
%
% ------------------------------------------------------------------------ %
% 	CHAPTER TITLE
% ------------------------------------------------------------------------ %
%
\chapter{Overall Description}
%
\label{cap:overalldescription}
%
% ------------------------------------------------------------------------ %
%
\section{Actors}
\begin{itemize}
\item Guest: Person that’s not registered yet. He can see the login and the registration page.
\item User: Person registered and logged to the app. He can access to his calendar to see his appointments, setup new meetings or manage them, change his preferences, …
\end{itemize}
%
% ------------------------------------------------------------------------ %
%
\section{Goals}
The aim of the system is to give the user the following functionalities:
\begin{itemize}
\item registration to the service and preferences set up;
\item anytime management of personal and mobility preferences;
\item create and schedule a new event choosing time and location;
\item edit previously added event data;
\item change travel options on automatically created trips to/from event;
\item mobility companies’ tickets purchase via built-in browser;
\item provide high customizability of personal preferences regarding user daily routines.
\end{itemize}
and to provide on its own these other ones:
\begin{itemize}
\item manage in a clever way trips to/from user events, relying on the map of the surroundings, time of the day, public and private transports’ timetables and stops, shared means, traffic, weather forecast, possible strikes, event information and user preferences;
\item notify the user on time of upcoming trips;
\item warn him in case of an exception.
\end{itemize}
%
% ------------------------------------------------------------------------ %
%
\section{Domain Assumptions}
We suppose that the following conditions are true in the analysed world:
\begin{itemize}
\item the geographical area of the city is included in the coverage area of most common mobile communication technologies (3g, 4g) offered by main telecommunications companies;
\item users must be subscripted to a sharing service if they want to use it;
\item APIs used by the application will always be updated on traffic status, eventual incidents and weather conditions;
\item sharing services’ APIs signals their means if and only if the means are where the APIs say, and they are not occupied or booked;
\item users always have a working internet connection;
\item half an hour is enough warning time for users to start a trip;
\item the user cannot ride two means of transport at the same time.
\end{itemize}
%
% ------------------------------------------------------------------------ %
\section{Product Perspective}
Travlendar+ will be developed as a mobile application that relies on the use of Google maps and Google calendar APIs. \\
Its user interface will be composed by two main tabs, one with a calendar, to schedule user's events and the other one with a map to manage the movements of the user. \\
In the future the system will employ APIs to buy tickets without using the built-in browser and it will have a service of technical assistance via chat. \\
The application will not provide any API for integration with other systems.
%
% ------------------------------------------------------------------------ %
%
\section{User Characteristics}
The user of the system-to-be can be every person who wants to schedule appointments in a calendar and manage his movements from a location to another at the same time.
Users can use it to organize work events, but also include family or spare time events. The application doesn't have any age limit, or any other restriction applied to the user characteristic. In order to make the application work without limitation the user need to have access to the Internet, but he can access and modify the calendar offline.
%
% -----------------------------END------------------------------------- %