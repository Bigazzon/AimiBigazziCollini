% ------------------------------------------------------------------------ %
% !TEX encoding = UTF-8
% !TEX TS-program = pdflatex
% !TEX root = ../Project.tex
% !TEX spellcheck = en-EN
% ------------------------------------------------------------------------ %
%
% ------------------------------------------------------------------------ %
% 	CHAPTER TITLE
% ------------------------------------------------------------------------ %
%
\chapter{Introduction}
%
\label{cap:introduction}
%
% ------------------------------------------------------------------------ %
%
This document is the Requirement Analysis and Specification Document (RASD) of a mobile application called Travlendar+. The purpose of the document is to show the requirements and specification of the new application, considering various aspects like the stakeholders' needs, domain properties and constrains which the system-to-be is subject to.
%
% ------------------------------------------------------------------------ %
%
\section{Description of the given problem}
Travlendar+ is a mobile, calendar-based application that helps the user to manage his appointments and to a greater extent set up the trip to his destination, choosing the best means of transport depending on his needs. \\
Travlendar+ will choose the most suitable way to get the user to his destination between a large pool of options, considering public transportation, personal vehicles, locating cars or bikes of sharing services and walking to the destination. It will take account of weather, traffic, possible passengers if any, the user-set break times and the potential will to minimize the carbon footprint of the trip, always focusing on taking him on time to his scheduled appointments. \\
Eventually the user will be able to purchase the tickets he will use to reach his destination in-app. The great customizability is one of the main strengths of Travlendar+, being able to fully comply with the user needs. 
%
% ------------------------------------------------------------------------ %
%
\section{Definitions and Acronyms}
%
\subsection{Definitions}
\begin{itemize}
\item Application: it depends on context, but often it is referred to Travlendar+;
\item User: see 2.A - Actors;
\item Guest: see 2.A - Actors;
\item System: see Application;
\item Event: often used as a key word referring to the events the user can add on the calendar embedded in Travlendar+;
\item Trip: often used as a key word referring to the trips alternatives the system processes and suggests the user;
\item Dynamic event: Event whose location in time can be decided by the application on various factors;
\item Means of transport: often used as a key word referring to the means of transportation the user can choose in the settings of Travlendar+;
\item Private transports: often used to mean trains, planes and all private companies transports;
\item Shared means: used to mean cars and bicycles of car and bike sharing companies.
\end{itemize}
%
\subsection{Acronyms}
List of the acronyms used in this paper:
\begin{itemize}
\item RASD: Requirements analysis and specification document;
\item ETA: Estimated time of arrival, it's the time remaining to arrive to destination;
\item POI: Point of interest;
\item API: Application programming interface;
\item UI: User Interface.
\end{itemize}
%
% ------------------------------------------------------------------------ %
%
\section{Revision History}
\begin{itemize}
\item 29/10/2017 Version 1.0.0 - First complete drawing up of the RASD;
\item 18/11/2017 Version 1.0.1 - Fixed some grammatical errors;
\item 24/11/2017 Version 1.1.0 - Changed Sections 3.B.4 - Communication interfaces and 3.F.3 - Security;
\item 26/11/2017 Version 1.1.1 - Added some acronyms and modified requirement 34.
\end{itemize}
%
% ------------------------------------------------------------------------ %
%
\section{References}
Documents list:
\begin{itemize}
\item Mandatory Project Assignments.pdf
\item IEEE Std 830-1998 IEEE Recommended Practice for Software Requirements
Specifications
\item RASD sample from Oct. 20 lecture.pdf
\end{itemize}
Online sites list:
\begin{itemize}
\item http://alloy.mit.edu/alloy/documentation.html
\end{itemize}
%
% ------------------------------------------------------------------------ %
%
\section{Document Structure}
The paper is structured as follows:
\begin{itemize}
\item Chapter 1: Introduction to the document, including some information about its composition and the description of the problem;
\item Chapter 2: General description of the system functions and assumptions, together with information about the environment and the users;
\item Chapter 3: Detailed description of the functional, non functional requirements and constraints;
\item Chapter 4: System usage examples; 
\item Chapter 5: Specific modeling of system functions, abstract structure and usage;
\item Chapter 6: Formal analysis of the system and his functions using Alloy;
\item Chapter 7: Effort spent by the authors to draw up the document.
\end{itemize}
%
% ------------------------------------------------------------------------ %
%
\section{Used tools}
The tools used to create this document are:
\begin{itemize}
\item StarUML for UML mode;
\item Alloy Analyzer 4.2 for proving consistency of the model and to challenge it to find counterexamples for our assertions;
\item Github as version controller and to share documents;
\item LaTeX for typesetting this document;
\item Texmaker as editor
\item Draw.io to create sequence diagrams and state diagrams;
\item Signavio to create use case diagrams;
\item Photoshop for mockups.
\end{itemize}
%
% -----------------------------END------------------------------------- %
